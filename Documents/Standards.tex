\documentclass[a4paper,12pt]{article}
\usepackage{a4wide}
\usepackage{acronym}
\usepackage{graphicx}
\usepackage[title,titletoc,toc]{appendix}


\begin{document}

\begin{titlepage}
\begin{center}
\vspace*{3cm}
{\LARGE\bf Standards Plan}\\
\vspace{1.5cm}
{\large\bf for}\\
\vspace{1.5cm}
{\LARGE\bf The Java Draw Project}\\
\vspace{5cm}
Prepared by William Reid\\
\vspace{1cm}
School of Computer Science,\\
The University of Adelaide\\
\vspace{1cm}
\today
\end{center}
\end{titlepage}
%\thispagestyleempty

\section{Quality and Readability Standards}
\subsection{File log conventions}
Inherent in a large system is a large collection of files. As to maintain consistency, each file must contain its own header of comments. The following details the comment conventions.

Format:\\
/*\\
\hspace*{1.5pt} \** Author(s)\\
\hspace*{1.5pt} \** Date created\\
\hspace*{1.5pt} \** Sub-system\\
\hspace*{1.5pt} \** Date/Time updated\\
\hspace*{1.5pt} \** Purpose\\
\hspace*{1.5pt} \**/

\begin{enumerate}
\item Author: Name \& Student ID
\item Date created: Date file created
\item Sub-system: Short description of what sub-system it is a part of
\item Date/Time updated: Date file most recently updated
\item Purpose: Short description of the files purpose\\*
\end{enumerate}

\subsection{File log conventions}
To maintain consistency, readability and maintainability, each function must contain its own set of comments above its declaration. The following details the comment conventions.\\

Format:\\
/**\\
\hspace*{1.5pt} \** Purpose\\
\hspace*{1.5pt} \** Short explanation\\
\hspace*{1.5pt} \** @param - Parameter type and an explanation of its purpose (new line for each parameter)\\
\hspace*{1.5pt} \** @return - Return type and explanation of why it is being returned\\
\hspace*{1.5pt} \**/


\newpage
\subsection{Function/Variable Naming}
To further ensure consistency and readability, each function and variable name will follow a simple convention explained below:

\subsubsection{Function Names}
\begin{enumerate}
\item Will use camel case i.e. checkForFiles()
\item Must not be abbreviated i.e. avg must be average
\end{enumerate}

\subsubsection{Variable Names}
\begin{enumerate}
\item Will use underscores rather than camel case i.e. brightness\_value
\item Must not be capitalised, but may contain capitals where applicable
\item Must not be abbreviated i.e. avg must be average
\end{enumerate}



\end{document}
